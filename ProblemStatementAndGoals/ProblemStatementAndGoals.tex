\documentclass{article}
\usepackage{geometry}
\geometry{a4paper, margin=1in}
\usepackage{amsmath}

\title{Problem Statement}
\begin{document}

\section*{Problem Statement}

The current McMaster GSA softball league platform is outdated and requires extensive programming knowledge for administrators to manage league operations, making it difficult for non-technical users to effectively maintain the system. This results in inefficient scheduling, score tracking, and rescheduling processes, which can lead to delays and communication breakdowns between administrators, captains, and participants. Additionally, the lack of an intuitive, user-friendly interface limits the ability for league representatives to perform essential tasks, such as waiver management, without technical support, hindering overall league management and participant experience.

\section*{Project Goals}

\subsection*{1. User-Friendly League Management Platform}
Develop a web-based platform that allows non-technical users, such as administrators and team representatives, to easily manage all aspects of the league without requiring programming knowledge. The platform should have intuitive navigation and role-specific access to facilitate operations like scheduling, score tracking, waiver management, and communication. \\
\textbf{Measurable Goal}: Achieve a user satisfaction rating of at least 85\% based on post-implementation surveys with league administrators and captains.

\subsection*{2. Automated Scheduling and Standings System}
Implement an automated scheduling feature that dynamically updates schedules, handles rescheduling requests, and automatically calculates and displays league standings based on match outcomes. \\
\textbf{Measurable Goal}: Reduce the time required to create and update schedules by 75\%, verified by comparing the new platform with the previous system.

\subsection*{3. Streamlined Communication Tools}
Integrate a communication system within the platform to allow direct and efficient communication between administrators and team captains, including notifications for important updates, match reschedules, or waiver completions. \\
\textbf{Measurable Goal}: Ensure that 100\% of all notifications related to scheduling or administrative updates are successfully delivered within 1 minute.

\subsection*{4. Secure Waiver Management}
Provide a secure, digital waiver management system where participants can electronically sign and submit waivers, and administrators can easily track and verify completion. The system should store waivers in compliance with relevant privacy and security standards. \\
\textbf{Measurable Goal}: Ensure 100\% compliance with privacy policies and securely manage all participant waiver data.

\subsection*{5. Role-Based Access Control}
Implement role-based access control to ensure that different users, such as league administrators and team captains, can access only the information and functionalities relevant to their responsibilities. This will minimize administrative errors and safeguard sensitive league data. \\
\textbf{Measurable Goal}: Verify that role-based permissions prevent 100\% of unauthorized access attempts.\\\\

\section*{Stretch Goals}

\subsection*{6. Mobile-Friendly Interface}
Design a mobile-responsive version of the platform so that users can access and manage league functionalities from their smartphones. This feature would add convenience for participants and administrators who need to interact with the system on the go. \\
\textbf{Measurable Goal}: Ensure that the mobile interface has at least 80\% feature parity with the desktop version, verified through feature testing and user feedback.

\subsection*{7. Predictive Scheduling with Machine Learning}
Explore the use of machine learning algorithms to optimize scheduling by predicting potential conflicts, such as weather conditions or team availability, and automatically proposing alternative schedules to minimize disruptions. \\
\textbf{Measurable Goal}: Achieve a conflict prediction accuracy of 85\%, verified by comparing predicted and actual conflicts.

\end{document}
